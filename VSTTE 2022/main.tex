% This is samplepaper.tex, a sample chapter demonstrating the
% LLNCS macro package for Springer Computer Science proceedings;
% Version 2.20 of 2017/10/04
%
\documentclass[runningheads]{llncs}
%
\usepackage{graphicx}
\usepackage{url}
\usepackage{xspace}
\usepackage{paralist}
\usepackage{subfiles}



% Used for displaying a sample figure. If possible, figure files should
% be included in EPS format.
%
% If you use the hyperref package, please uncomment the following line
% to display URLs in blue roman font according to Springer's eBook style:
% \renewcommand\UrlFont{\color{blue}\rmfamily}
\newcommand\hyperedgeGNN{R-HyGNN\xspace}
\newcommand\hyperedgeGNNs{R-HyGNNs\xspace}


\begin{document}
%
\title{Contribution Title\thanks{Supported by organization x.}}
%
%\titlerunning{Abbreviated paper title}
% If the paper title is too long for the running head, you can set
% an abbreviated paper title here
%
\author{Chencheng Liang\inst{1}%\orcidID{0000-0002-4926-8089} 
\and
Philipp Rümmer\inst{1,2}%\orcidID{0000-0002-2733-7098} 
\and
Marc Brockschmidt\inst{3}%\orcidID{0000-0001-6277-2768}
}
%
\authorrunning{C. Liang et al.}
% First names are abbreviated in the running head.
% If there are more than two authors, 'et al.' is used.
%
\institute{
Uppsala University, Uppsala, Sweden\\
%Department of Information Technology
\email{chencheng.liang@it.uu.se}\\
\and
University of Regensburg, Regensburg, Germany\\
\email{philipp.ruemmer@it.uu.se}\\
\and
Microsoft Research\\
\email{marc@marcbrockschmidt.de}
}




%
\maketitle              % typeset the header of the contribution
%
\begin{abstract}
The abstract should briefly summarize the contents of the paper in
15--250 words.

\keywords{Automatic program verification  \and Constraint Horn clauses \and Graph neural networks.}
\end{abstract}
%
%
%
% \subfile{sections/1-introduction}
% \subfile{sections/2-background}
% \subfile{sections/3-template-selection}
% \subfile{sections/4-evaluation}
% \subfile{sections/5-related-works}
% \subfile{sections/6-conclusion-and-future-works}


\import{sections/}{1-introduction}
\import{sections/}{2-background}
\import{sections/}{3-CEGAR}
\import{sections/}{4-template-selection}
\import{sections/}{5-evaluation}
\import{sections/}{6-related-works}
\import{sections/}{7-conclusion-and-future-works}


\bibliographystyle{splncs04}
\bibliography{mybibliography}

\end{document}
